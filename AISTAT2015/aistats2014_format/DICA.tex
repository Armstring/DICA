\documentclass[twoside]{article}
\usepackage{aistats2014}
\usepackage{amsthm}
\usepackage{amsmath}
\usepackage{amssymb}
\usepackage[authoryear]{natbib}
\usepackage{enumerate}
\usepackage{algorithmic}
\usepackage{algorithm}
\usepackage{array}

\newcommand{\xcom}[1]{x_{#1}}
\newcommand{\scom}[1]{s_{#1}}
\newcommand{\cset}[2]{\left\{#1\,:\,#2\right\}}
\newcommand{\Ephione}{\mathbb{E}_{\phi_1}}
\newcommand{\Ephitwo}{\mathbb{E}_{\phi_2}}
\newcommand{\Epsi}{\mathbb{E}_{\psi}}
\newcommand{\cN}{\cal{N}}
\renewcommand{\P}{{\mathcal P}}
\newcommand{\E}{\mathbb{E}}
\newcommand{\Ex}[1]{\mathbb{E}[#1]}
\newcommand{\Em}[2]{\mathbb{E}_{#1}\left[#2\right]}
\newcommand{\Prob}[1]{\mathbb{P}\left(#1\right)}
\newcommand{\Var}{\mathrm{Var}}
\newcommand{\tr}{\mathrm{tr}}
\newcommand{\norm}[1]{\|#1\|}
\newcommand{\snorm}[1]{\left\|#1\right\|} % scaling norm
\newcommand{\lmax}[1]{\lambda_{\mathrm{max}}(#1)}
\newcommand{\lmin}[1]{\lambda_{\mathrm{min}}(#1)}
\newcommand{\sign}{\mathrm{sign}}
\newcommand{\dprod}[2]{\langle #1,#2 \rangle_{M}}
\newcommand{\hA}{\hat{A}}
\newcommand{\hb}{\hat{b}}
\newcommand{\hC}{\hat{C}}
\newcommand{\hAp}{\hat{A}^\prime}
\newcommand{\hbp}{\hat{b}^\prime}
\newcommand{\hCp}{\hat{C}^\prime}
\renewcommand{\th}{\theta}
\newcommand{\vth}{\hat{\theta}_{V}}
\newcommand{\lth}{\hat{\theta}_{\lambda}}
\newcommand{\hth}{\hat{\theta}}
\newcommand{\ltho}{\hat{\theta}_{\lambda_1}}
\newcommand{\ltht}{\hat{\theta}_{\lambda_2}}
\newcommand{\hlth}{\hat{\theta}_{\hat{\lambda}}}
\newcommand{\slth}{\hat{\theta}_{\lambda^*}}
\newcommand{\plth}{\hat{\theta}_{\lambda_p}}
\newcommand{\hthp}{\hat{\theta}^\prime}
\newcommand{\lthp}{\hat{\theta}^\prime_{\lambdap}}
\newcommand{\ath}{\hat{\theta}_{a}}
\newcommand{\thetap}{\theta^\prime}
\newcommand{\sth}{\theta^*}
\newcommand{\ind}[1]{\mathbb{I}_{\left\{ #1 \right\}}}
\newcommand{\hzeta}{\hat{\zeta}}
\newcommand{\ie}{\emph{i.e.}}
\newcommand{\eg}{\emph{e.g.}}
\newcommand{\cf}{\emph{cf.}}
\newcommand{\T}[1]{T\left( #1 \right)}
\newcommand{\St}[1]{S\left( #1 \right)}
\newcommand{\Ts}[2]{T_{#1}\left( #2 \right)}
\newcommand{\Ss}[2]{S_{#1}\left( #2 \right)}
\newcommand{\Ord}[1]{O\left( #1 \right)}
\newcommand{\etc}{\emph{etc.}}
\newcommand{\bsth}{\bar{\theta}^*}
\newcommand{\sal}{\emph} %Emphasis the J.D. Salinger way :)
\newcommand{\gradL}{\nabla \mathcal{L}}
\newcommand{\lambdap}{{\lambda^\prime}}
\newcommand{\lambdahp}{\hat{\lambda}^\prime}
\newcommand{\pqLoss}[1]{\mathcal{L}_{#1}}
\newcommand{\pLoss}[2]{\pqLoss{#1}( #2 )}
\newcommand{\qLoss}{\pqLoss{M}}
\newcommand{\Loss}[1]{\qLoss(#1)}
\newcommand{\hpqLoss}[1]{\mathcal{\hat{L}}_{#1}}
\newcommand{\hpLoss}[2]{\hpqLoss{#1}( #2 )}
\newcommand{\hqLoss}{\hpqLoss{M}}
\newcommand{\hLoss}[1]{\hqLoss(#1)}
\newcommand{\bX}{\bar{X}}
\newcommand{\bY}{\bar{Y}}
\newcommand{\rth}{\hat{\theta}}
\newcommand{\tth}{\tilde{\theta}}
\newcommand{\maxeig}{\nu_{\max}}
\newcommand{\mineig}{\nu_{\min}}
\newcommand{\ra}{\rightarrow}
\newcommand{\real}{\mathbb{R}}
\newcommand{\one}[1]{\mathbf{1}_{\{#1\}}}
\newcommand{\rl}[1]{\mathbb{R}^{#1}}
\newcommand{\Mset}{\mathcal{M}(\varepsilon)}
\newcommand{\Cset}{\mathcal{C}}
\newcommand{\truel}{L}
\newcommand{\zol}{L_{\mathrm{0-1}}}
\newcommand{\hinl}{L_{\mathrm{hinge}}}
\newcommand{\phil}{L_{\varphi}}
\newcommand{\lspace}{\left\{1, \dots,K \right\}}
\newcommand{\risk}{\mathcal{R}}
\newcommand{\scoref}{\in \mathcal{H}}
\newcommand{\subscoref}{\in \mathcal{H}^\prime}
\newcommand{\hQ}{\hat{Q}}
\DeclareMathOperator{\supp}{supp}
\DeclareMathOperator{\esssup}{ess\,sup}
\renewcommand{\natural}{\mathbb{N}}


% theorems/definitions
\newtheorem{lemma}{Lemma}[section]
\newtheorem{thm}[lemma]{Theorem}
\newtheorem{claim}[lemma]{Claim}
\newtheorem{cor}[lemma]{Corollary}
\newtheorem{example}[lemma]{Example}
\newtheorem{prop}[lemma]{Proposition}
\theoremstyle{definition}
\newtheorem{definition}[lemma]{Definition}
\newtheorem{remark}[lemma]{Remark}
\newtheorem*{solution}{Solution}
\newtheorem{note}[lemma]{Note}
\newtheorem{problem}[lemma]{Problem}

\newcommand{\FF}{\mathcal{F}}
\newcommand{\TT}{\mathcal{T}}
\renewcommand{\AA}{\mathcal{A}}
\newcommand{\KK}{\mathcal{K}}
\newcommand{\eps}{\varepsilon}

% If your paper is accepted, change the options for the package
% aistats2014 as follows:
%
%\usepackage[accepted]{aistats2014}
%
% This option will print headings for the title of your paper and
% headings for the authors names, plus a copyright note at the end of
% the first column of the first page.


\begin{document}

% If your paper is accepted and the title of your paper is very long,
% the style will print as headings an error message. Use the following
% command to supply a shorter title of your paper so that it can be
% used as headings.
%
%\runningtitle{I use this title instead because the last one was very long}

% If your paper is accepted and the number of authors is large, the
% style will print as headings an error message. Use the following
% command to supply a shorter version of the authors names so that
% they can be used as headings (for example, use only the surnames)
%
%\runningauthor{Surname 1, Surname 2, Surname 3, ...., Surname n}

\twocolumn[

\aistatstitle{Deterministic Independent Component Analysis}

\aistatsauthor{ Anonymous Author 1 \And Anonymous Author 2 \And Anonymous Author 3 }

\aistatsaddress{ Unknown Institution 1 \And Unknown Institution 2 \And Unknown Institution 3 } ]

\begin{abstract}
Abstract.
\end{abstract}

\section{Introduction}
\label{sec:Intro}
Independent Component Analysis (ICA), as a main tool for separating blind sources, has received much attention in the past decades. 
The ICA model assumes a $d$-dimensional vector $X$ is a linear mixture of $d$ independent variables $(S_1,\ldots, S_d)$ with Gaussian noise:
\[
X = AS+\epsilon,
\]
where $\epsilon$ is a $d$-dimensional Gaussian noise, and $A$ is a nonsingular mixing matrix.
Our interest is to reconstruct $A$ given observations of $X$ with rigorous guarantee.

The literature of the ICA model is vast, varying from parametric models to semi-parametric ones
\citep{comon1994independent,pham1997blind,lee1999independent,HyvOja00,samarov2004nonparametric,eriksson2003characteristic,chen2005consistent,chen2006efficient}. 
In the theoretical side, \citet{samarov2004nonparametric} for the fist time, to our best knowledge, proposed a $\sqrt{n}$-consistent estimator for $A$ in a noise-free setting (i.e. $X= AS$). 
Then other more efficient estimators are proposed later in \citep{chen2005consistent,chen2006efficient}. 
In the noisy setting, the first algorithm with rigorous performance guarantee is attributed to \cite{arora2012provable}. 
Recently as the progress that has been made in tensor decomposition, tensor-based algorithms are developed, rigorously analyzed, and becoming more and more popular because of its user friendliness in practice \citep{hsu2013learning,anandkumar2012tensordecomposition,anandkumar2014provable,goyal2014fourier}.  
A more detailed discussion about the ICA literature is introduced later in Subsection \ref{subsec:RelatedWorks}.

Despite all the progresses that have been made, all the present works are conducted in some stochastic setting.
However, in practice ICA is also known to work well for unmixing the mixture of various deterministic signals. 
One of the classic demonstrations of ICA is showing that the periodic signals can be well recovered from their mixtures. \citep{HyvOja00}.
Such phenomenon suggests that the usual probabilistic notion is unsatisfactory if one wishes to have deeper understanding of ICA.   
The goal of this paper is to propose a deterministic framework to investigate this curious phenomenon. 

Formally, let $s:\natural \ra \real^d$ be a $d$-dimensional deterministic ``signal''. We denote by $s_i$ the $i$th component of $s$.
The ICA model is presented as a linear mixture of functions (details will be introduced later).
We then define the `independence' for functions (sequences) based on their induced probability, which is a fundamental notion in the ICA model.
Compared to the traditional stochastic setting, our deterministic framework is more general to cover not only stochastic signals but also deterministic ones. 
In fact, Most of the results that hold in the stochastic setting should also go through in our deterministic framework.
We prove a same identifiability result as that in the stochastic setting to demonstrate this claim.

Moreover, concentration property of the random samples in the stochastic setting of ICA model is now replaced by deterministic conditions in terms of `functional distance'.
Such replacement provides us a deeper understanding about the ICA model and help separate the sampling randomness and the randomness of the algorithm itself in the theoretical analysis of ICA methods.
We take the ICA algorithm of \citep{hsu2013learning} as an example in our paper.
A detailed analysis shows that its performance depends on the minimal gap of the eigenvalues (of some random matrix. The exact form is presented later).
Unfortunately we have difficult in achieving any meaningful lower bound (polynomial in the parameters of the ICA model) for this minimal gap.
The lack of such lower bound essentially impair the soundness of the algorithm. 
Another contribution of this paper is a refined provably polynomial-time algorithm that avoids the minimal gap problem. 
We will discuss the most related ICA works in the literature and then briefly introduce our algorithm in Subsection \ref{subsec:RelatedWorks}.

The rest of this paper is organized in this way: 
in Section \ref{sec:DeterFrame} we will propose our deterministic framework. 
Then an identifiability result framework is presented in Subsection \ref{subsec:Indentifiability}.
We analyze the algorithm of \cite{hsu2013learning} and discussed the exponential dependence problem in Subsection \ref{subsec:HKalg} and \ref{subsec:errorHK}. 
Based on the above discussion, our algorithm is proposed in Subsection \ref{subsec:DICA} and its theoretical result is presented in Subsection \ref{subsec:gammaR}.
Lastly, simulation results are reported in Section \ref{sec:ExpRes}.

\subsection{Related Works}
\label{subsec:RelatedWorks}
One class of ICA methods is to find a linear transformation $W$ for $X$, such that the resulted coordinates of $WX$ are as independent as possible \citep{comon1994independent,HyvOja00}.
The optimal $W$ would be $A^{-1}$ under additional assumptions about the sources, thereby $A$ is recovered. 
An ICA method following this idea would maximize some designed contrast function measuring likelihood or negative mutual information, under the assumption of a particular parametric model \citep{comon1994independent,pham1997blind,lee1999independent,HyvOja00}. 
Typically theoretical analysis of these methods are limited, due to the difficulty in understanding the statical properties of those proposed contrast functions[********].  

Another class of ICA methods is viewing ICA in a semiparametric manner with parameters $(W, p_1, \ldots, p_d)$ where $W$ is the targeted matrix, and nuisance parameters $p_i$ is the density function of the $i$-th source. 
Under additional statistical assumptions about the source signals, consistent estimators have been proposed for various settings. The first  $\sqrt{n}$-consistent estimator of $A$, to our best knowledge, is proposed and analyzed in \citet{samarov2004nonparametric}.  
Later \citep{chen2005consistent} showed that the estimator proposed in \citet{eriksson2003characteristic} is also $\sqrt{n}$-consistent. Another estimator which is more efficient is also proposed by the same authors in \citep{chen2006efficient}.
However, all these results can only work for a noise-free setting (i.e. $X = AS$). 
Also, they all involves a intermediate procedure of estimating a probability function or a score function, which consequentially invites additional technical conditions so that the consistency could be maintained. 

The first provable ICA method that can deal with the noisy case is by \citep{arora2012provable}. After a quasi-whitening procedure, \citep{arora2012provable} reduced the ICA problem into a problem of finding all the local optimizer of a specific function defined using the forth order cumulant. Then a polynomial-time algorithm is proposed to find all the local optimizers with theoretical guarantee.
Despite its great progress in theory, \citep{arora2012provable}'s algorithm involves a tuning procedure for an input parameter ($\beta$ in the paper) that is not well understood, which impairs its usage in practice.

The forth moment statics are also used in many other ICA related works. 
Recently, a series of more promising methods which manipulates the forth order moment are proposed based on the idea of moment method \citep{hsu2013learning,anandkumar2012tensordecomposition,anandkumar2012method}. 
\citep{hsu2013learning} firstly apply the moment method to the ICA problem. The algorithm is claimed to have provable performance without rigorous proof. 
This idea is then greatly developed by \citep{anandkumar2012tensordecomposition,anandkumar2012method,goyal2014fourier} in a general tensor perspective of the moment method. 
These works are all well analyzed using matrix perturbation tools. 
However, the results exponentially depend on the number of source signals $d$.
Note that these methods all require an eigen-decomposition of some flatten tensor where the minimal gap between the eigenvalues plays an essential role. 
A naive analysis of this gap will introduce an exponential dependence on the dimension of the flatten tensor $d$. 
This dependence is also observed in Theorem 4.11 of \citep{goyal2014fourier}.
One way to circumvent such dependence would be directly decomposing a high order tensor using power method which requires no flatten procedure, as in \citep{anandkumar2014guaranteed}. 
However, it is well known that power method is unstable in practice for high order tensor. 
Another drawback is when applied to the ICA problem, this method introduces a constant term in its theoretical result which does not approach 0 as the sample size approaches infinity. 

In this paper we proposed a practical method for ICA which has meaningful theoretical guarantee(polynomial-time; with constant probability the reconstruction error of $A$ polynomially depends on all the parameters and vanishes as the sample size going to infinity).      
A key step of our algorithm is reducing the minimal gap problem as the problem of minimal spacing of i.i.d. Cauchy random variables, for which a polynomial lower bound is available.
This reduction is by the quasi-whitening procedure, which is also used in \citep{frieze1996learning} and \citep{arora2012provable}.
With such lower bound, we then show that the new algorithm has provable and meaningful guarantee on the reconstruction error of $A$. 
Our algorithm does not need any parameter tuning.

\subsection{Notations}
Denote the maximal (respectively minimal) singular value of a matrix $A$ by  $\sigma_{\max}(A)$ (respectively $\sigma_{\min}(A)$). Also, let $A_{(2,\min)} = \min_{i} \|A_i\|_2$, $A_{(2,\max)} = \max_{i} \|A_i\|_2$, and $A_{\max} = \max_{i,j} |A_{i,j}|$.

\section{Deterministic Framework}
\label{sec:DeterFrame}
%In this section we will first define a natural notion of independence for functions. Then a deterministic framework is proposed based on the defined `independence'. 

Assume we are given a $d$-dimensional observed function $x(t)$ defined by  
\begin{equation}
\label{equ:ICA}
x(t) = As(t), \quad t\ge0
\end{equation}
where $A$ is a $d\times d$ nonsingular matrix and  $s$ is a $d$-dimensional signal function. 
The goal of the present paper is to investigate under what conditions $A$ can be reconstructed(with `small' error in polynomial time).

Note that the above deterministic setting does not lose any generality to the traditional stochastic setting of ICA with Gaussian noise, given that we can rewrite the stochastic setting as 
\[
x = As+\eps = A(s+A^{-1}\eps),
\] 
where $\eps$ is the Gaussian noise. In practice, we are not going to observe the true distribution of $x$, but $x(t)$, $t\ge 0$, which leads to Equation \eqref{equ:ICA}.

Clearly, no meaningful results will be achieved if $s$ is allowed to go arbitrarily wild. 
In the stochastic setting, ICA assumes the independence of the signals and the Gaussianity of the noise. 
In this section, we propose a natural notion fot the independence of signal functions in an analogous way, which is compatible with the stochastic setting but much more general.

\subsection{Independent Components}
\label{subsec:IndeComp}
Given arbitrary measurable domains $(\Omega,\AA)$ and a sequence of probability measures $(\mu_t)_{t\ge0}$ over it, we define $\nu_t$ to be the probability measure over $\real^d$ that is induced by $s$ and $\mu_t$ for any function $s: \Omega \ra \real^d$ and $t\ge 0$:
In particular, for any Borel set $A\subset \real^d$,
\[
\nu_t(A)= \mu_t\Bigl( \cset{z}{s(z)\in A} \Bigr)
= \mu_t ( s^{-1}(A) ).
\]
Note that if $\Omega = \natural$, $\AA = 2^\natural$ and $\mu_t$ is the uniform probability distribution on $\{0,\ldots,t\}$, then $\nu_t$ is the empirical probability.

In what follows we fix $\Omega$ and $(\mu_t)$ and define all the concepts that follow relative to these. 
In this paper we only consider about real functions.
When the limit of $(\nu_t)$ exists, it will be denoted by $\nu$.
Also, if we want to emphasize the dependence on $s$ then we will use $\nu^{(s)}$.
Similarly, when the dependence of $\nu_t$ on $s$ is important, we will use $\nu^{(s)}_t$.


\begin{definition}
A function $s:\Omega \rightarrow \real^d$ is called \emph{ergodic} w.r.t. $(\mu_t)_{t\ge0}$
if the sequence of  induced probability measures $(\nu_t)_{t\ge 0}$ is weakly convergent.
\end{definition}
A simple example of an ergodic function is a real periodic function.
The following lemma shows that once $\nu$ exists, it is a probability measure. 
\begin{prop}
\label{lem:ergodicfunction}
Let $\scom{i}$ denote the $i$th coordinate function of $s$. Assume that
\begin{equation}
\label{eq:ergodicproperty}
\lim_{t\to\infty} \int |\scom{i}(x)|\, d\mu_t(x) 
\end{equation}
exists and is finite for all $1 \le i \le d$ and that $\esssup_{\mu_t} |\scom{i}|<\infty$ for any $1\le i \le d$ and $t\ge 0$.
Then all limit points of the sequence $\{\nu_t\}$, with respect to the weak topology, are probability measures over $\real^d$.
\end{prop}

Despite the perceptual intuitiveness of the definition of ergodic function, the space of ergodic functions is, unexpectedly, not closed under addition, as shown in Example \ref{eg:ergodic}. 
We provide a sufficient and necessary condition for a linear combination 
of ergodic functions to be ergodic in Proposition \ref{prop:comp}. 
It turns out the space spanned by the components of an ergodic function is a space of ergodic functions.
Such observation leads to our definition of independence of components, and build the foundation of the rest work of our paper.
\begin{example}
\label{eg:ergodic}
Let $f,g:\natural \rightarrow \{-1,+1\}$ and $(\mu_t)$ be defined as follows: 
\[
f(t) = \begin{cases} +1, & t \equiv 0 \pmod 4; \\
					-1, & t \equiv 1 \pmod 4; \\
					+1,  & t \equiv 2  \pmod 4; \\
					-1, & t \equiv 3  \pmod 4,
					\end{cases}
					\]\[
g(t) = \begin{cases} +1, & t\equiv 0 \pmod 4; \\
					-1, & t\equiv 1 \pmod 4; \\
					-1,  & t\equiv 2 \pmod 4; \\
					+1, & t\equiv 3 \pmod 4.
					\end{cases}
\]
\[
\quad \mu_{2t+1}(x) = \begin{cases} 1/2, & x=0; \\
					1/2, & x=1, 
					\end{cases}
\mu_{2t}(x) = \begin{cases} 1/2, & x=2; \\
					1/2, & x=3;.
					\end{cases}
\]
Then, $\nu^{(f)}_t$, $\nu^{(g)}_t$ are both the uniform  distribution on $\{-1,+1\}$, hence they converge and the limit is the same uniform  distribution. 
However, while $\nu^{(f+g)}_{2t+1}$ is the uniform  distribution on $\{-2,+2\}$,  $\nu^{(f+g)}_{2t}$ is the degenerate
distribution that puts all the mass at $0$. Thus, $\nu^{(f+g)}_{t}$ fails to be convergent. 
\end{example}

\begin{prop}\label{prop:comp}
$s = (\scom{1},\ldots,\scom{d})^{\top}$ is a $d$-dimensional ergodic function, if and only if 
 for any $m\times d$ matrix $A$, $x = A s$ is an $m$-dimensional ergodic function.
\end{prop}  
Based on Lemma \ref{prop:comp}, we can now define the independence of the components of a multidimensional ergodic function $f$ based on its induced limiting distribution $\nu^{(f)}$:
\begin{definition}
Given an ergodic function $s = (\scom{1},\ldots, \scom{d})^{\top}:\Omega \rightarrow \real^d$, 
we say that $\{\scom{1},\ldots, \scom{d}\}$ are \emph{independent} components, 
	if $\nu^{(\scom{1})}\otimes\ldots\otimes\nu^{(\scom{d})} = v^{(s)}$.
\end{definition}
\begin{remark}
Note that the existence of $\nu^{(s^{(i)})}$ is guaranteed by Proposition \ref{prop:comp}. 
The definition of independent components can also be extended to the independence of ergodic functions: 
$f$ and $g$ are called independent, if their joint function $(f,g)$ is an ergodic function with independent components. 
\end{remark}
\begin{remark}
Assume that $f = (f_1,f_2)^{\top}:\real\rightarrow\real^2$ is an ergodic function, and both $f_1$ and $f_2$ are periodic with respective periods $T_1$ and $T_2$. 
Let $\mu_t$ be the uniform probability distribution on $[0,t]$. 
%$\mu_t((a,b)) = (b-a)/t$ for $0\le a\le b\le t$.
 If $T_1/T_2$ is irrational, then it is easy to show that $f_1$ and $f_2$ are independent. 
 This example shows how period signals can be fitted into our framework.
\end{remark}

\section{Results}
\label{sec:Results}
We now analyze the ICA problem in our deterministic framework proposed in Section \ref{sec:DeterFrame}.

\subsection{Identifiability}
\label{subsec:Indentifiability}
The first essential question is the identifiability of the ICA model.
Our result requires the minimal assumption that each component $s_i$ is non-Gaussian\footnote{this result could be extended to allow one Gaussian.}, which is the same as that of stochastic setting.
\begin{thm}
\label{thm:CorofICA}
Let $s = (\scom{1},\ldots,\scom{d})^{\top}$ be an ergodic function with independent components and
assume that neither of $\nu_{\scom{i}}$, $i=1,\ldots,d$ is a Gaussian distribution. If
\begin{equation}
\left(
\begin{array}{ccc}
\xcom{1} \\
\vdots \\
\xcom{d}
\end{array}
\right) = A
\left(
\begin{array}{ccc}
\scom{1} \\
\vdots \\
\scom{d}
\end{array}
\right)
\end{equation}
are independent, then $A$ is equivalent to a permutation matrix up to scaling.
\end{thm}

\subsection{Reconstruction Error}
\label{subsec:HowDifficult}

Standard results about the reconstruction of $A$ can be phrased as follows:
Let $S$ be a random $d$-dimensional vector, whose components are independent of each other.
Let $X = A S$, where $A$ is a nonsingular $d\times d$ matrix. Then, given the empirical distribution of $X$, assuming that the joint of $S$ satisfies some additional assumptions, the matrix $A$ can be approximately reconstructed up to permutation and scaling, with provable error.

In general, we believe these results carry through to the deterministic setting, by replacing the empirical distribution of $S$ with $\nu^{(s)}$ where $s$ is a $d$-dimensional deterministic signal.
In this paper, we analyze a randomized ICA algorithm of \citep{DHsu2012}(HK algorithm) as an example. 
In the deterministic setting, it clearly shows that the sample procedure plays an important role in the performance.
However, we have difficult in analyzing the behavior of this sample procedure, which leads to a vacuous theoretical result.
To solve this problem, we proposed a refined version of the HK algorithm, called deterministic ICA algorithm. 
This is the first `real' polynomial time moment-based algorithm for ICA. 
Details will be discussed in the next section due to its complexity.  

\section{Deterministic ICA}
\label{sec:DICA}
We analyze and discuss the performance of \citet{hsu2013learning}'s algorithm. Then based on the discussion, we propose our algorithm as a refinement.

\subsection{\citet{hsu2013learning}'s algorithm (HK)}
\label{subsec:HKalg}
For $p\ge 1$, $\eta\in \real^d$, 
let 
\begin{equation}
\label{eq:mpdef}
m_p(\eta) = \E_{Y\sim \nu^{(s)}}[ (\eta^\top A Y)^p ]
\end{equation}
 and let
\begin{equation}\label{eq:fdef}
f(\eta) = \frac1{12} \left( m_4(\eta) - 3 m_2(\eta)^2 \right)\,.
\end{equation}
\citet{DHsu2012} proposed an algorithm as follows, and claimed without proof that it has a provable performance. 
In general, we denote by adding a ``hat'' on the top of a symbol its empirical estimate.
\begin{algorithm}[H]
\caption{HK algorithm}
\begin{algorithmic}[1]
\INPUT $g(k)$ for $0\le k \le T$. 
\OUTPUT An estimation of the mixing matrix $A$. 
\STATE Sample $\phi$ from the unit sphere of $\real^d$;
\STATE Evaluate $\nabla^2\widehat{f}(\phi)$ and $\nabla^2\widehat{f}(\psi)$, 
%\quad where $\widehat{m_p}(\eta) = \frac{1}{T}\sum_{k=1}^{T} (\eta^{\top}g(k))^p$, and $\widehat{f}(\eta) = \frac{1}{12}\big(\widehat{m_4}(\eta) - 3\widehat{m_2}(\eta)^2 \big)$;
\STATE Compute $\widehat{M} = (\nabla^2 \widehat{f}(\phi))(\nabla^2\widehat{f}(\psi))^{-1}$;
\STATE Compute all the eigenvectors of $\widehat{M}$, $\{\mu_1,\ldots,\mu_d\}$;
\STATE Return $\widehat{A} = (\mu_1,\ldots,\mu_d)$.
\end{algorithmic}
\end{algorithm}
In particular, $\nabla^2\widehat{f}(\eta)$  is evaluated from the observations by
\[
\nabla^2 \widehat{f}(\eta) = \widehat{G}(\eta):= \widehat{G_1}(\eta) - \widehat{G_2}(\eta) -2\widehat{G_3}(\eta),
\]
where 
\begin{align*}
&\widehat{ G_1}(\eta) = \frac1n\sum_{t=1}^{n} \big(\eta^{\top}x(t)\big)^2x(t)x(t)^{\top}; \\
& \widehat{G_2}(\eta) = \frac{1}{n^2}\sum_{t=1}^{n} \big(\eta^{\top}x(t)\big)^2 \sum_{t=1}^{n}x(t)x(t)^{\top}; \\
& \widehat{G_3}(\eta) = \frac{1}{n^2}\Big(\sum_{t=1}^{n} \big(\eta^{\top}x(t)\big)x(t)\Big) \Big(\sum_{t=1}^{n} \big(\eta^{\top}x(t)\big)x(t)\Big)^{\top}.
\end{align*} 

\subsection{Reconstruction error of HK algorithm}
\label{subsec:errorHK}
Assume the independent ergodic component functions $\scom{1},\ldots,\scom{d}:\natural \ra \real$ are bounded by a constant $C$, and satisfy $\E_{Y\sim\nu^{(\scom{i})}}[Y]=0$ and $\kappa_i := \E_{Y\sim \nu^{(\scom{i})}}[Y^4] - 3\left(\E_{Y\sim \nu^{(\scom{i})}}[Y^2]\right)^2\neq 0$.
%\footnote{In this section, we will denote the sources by $s_i$ to avoid a clash with the function $f$ used by  \citet{DHsu2012} for a different purpose.}
%Note that sample a vector $\psi$ from a unit sphere, $\min_i |\psi^{\top}A_i|\neq 0$ with probability~1.  
Denote the diagonal matrix $\text{diag}(\kappa_1,\cdots,\kappa_d)$ by $K$. 
Also, denote $\max_{i} \kappa_i$ by $\kappa_{\max}$ and $\min_{i} \kappa_i$ by $\kappa_{\min}$.

Although $A$ is identifiable under the conditions of Theorem \ref{thm:CorofICA}, in practice the unobserved signal function $s$ will never have independent components (especially in a noisy setting). 
Hence we still need a `dependence measure' for the components of $s$.
\begin{definition}
Given two distributions $F_1$ and $F_2$ in $\real^d$, let $D_k(F_1,F_2) = \sup_{f\in\mathcal{F}} |\int f(x)dF_1(x) - \int f(x)dF_2(x)|$, where $\mathcal{F}$ is the set of all monomials up to degree $k$.
\end{definition} 
The following lemma provides a lower bound for $\min_i |\psi^{\top}A_i|$.
\begin{lemma}
\label{lem:dmin}
With probability at least $1/4$, $\min_i |\psi^{\top}A_i| \ge \frac{\sqrt{2\pi}}{2d}A_{(2,\min)}$.  
\end{lemma}
Denote the event of Lemma \ref{lem:dmin} by $\Epsi$. Let 
\begin{equation}
\label{def:kappa}
\gamma =  \min_{i,j: i\neq j} \left\vert \left(\frac{\phi^{\top}A_i}{\psi^{\top}A_i}\right)^2 - \left(\frac{\phi^{\top}A_j}{\psi^{\top}A_j}\right)^2 \right\vert, 
\end{equation}
and 
\[
\xi = \left( 6C^2D_2(\nu, \nu_T) + D_4(\nu, \nu_T)\right).
\]
The performance of \citet{DHsu2012}'s algorithm depends on these parameters, as shown in the following theorem.

 \begin{thm}
 \label{thm:efficiency}
Let 
 \begin{align}
 Q = &\Big(\frac{4d^7A_{(2,\max)}^4A_{\max}^2\kappa_{\max}\sigma_{\max}^2(A) }{\pi\kappa^2_{\min}A^4_{(2,\min)}\sigma_{\min}^4(A)} \\
 & + \frac{2\sqrt{2}\sqrt{\pi}d^6A_{(2,\max)}^2A_{\max}^2\kappa_{\min}A^2_{(2,\min)}\sigma_{\min}^2(A)}{\pi\kappa^2_{\min}A^4_{(2,\min)}\sigma_{\min}^4(A)} \Big)
  \xi.
 \end{align}
 Assume the following conditions hold:
 \begin{enumerate}
 \item $\widehat{M}$ has distinct eigenvalues;
 \item $\gamma > 4\frac{\sigma_{\max}(A)}{\sigma_{\min}(A)} Q$
 \item $\min_{i,j:i\neq j} \|A_i - A_j\|_2 > \frac{8}{\gamma}\frac{\sigma_{\max}^2(A)}{\sigma_{\min}(A) } Q$;
 \item $\xi \le \frac{\sqrt{2\pi}\kappa_{\min}A^2_{(2,\min)}\sigma_{\min}^2(A)}{4d^6 A_{(2,\max)}^2A_{\max}^2}$.
  \end{enumerate}
 Then on the event $\Epsi$, there exists a permutation $\pi$ and constants $\{c_1,\ldots,c_d\}$, such that
 \[
  \max_{1\le k\le d}\| c_1\widehat{A}_{\pi(k)} - A_k\|_2 \le \frac{4}{\gamma} \frac{\sigma_{\max}^2(A)}{ \sigma_{\min}(A)}Q .
  \]
  and therefore,
 \[
 \sum_{k=1}^{d}\| c_1\widehat{A}_{\pi(k)} - A_k\|_2 \le \frac{4d}{\gamma} \frac{\sigma_{\max}^2(A)}{ \sigma_{\min}(A)}Q .
 \]
 \end{thm}
\begin{remark}
Note that the first condition holds with probability 1. Also, the other three conditions will hold when $\xi$ is small enough. However, condition 2 and 3 are rarely satisfied in practice. 
More discussion about this problem and weaker conditions are discussed in Section \ref{sec:ExpRes}. 
\end{remark}

\begin{remark}
Note that $\gamma$ is actually the minimal gap of the eigenvalues of $M$.
Consider perturbing a $2\times 2$ matrix $A = \text{diag}([a,b])$ where $a<b$ by $E = \text{diag}([\eps,0])$. 
The eigen-spaces of $A+E$ will stay as $[1,0]$ and $[0,1]$ for $\eps < b-a$, but then can be arbitrary vectors when $\eps = b-a$.  
Due to such discontinuity of the eigen-spaces, the dependence on $\gamma$ is necessary once the algorithm takes the eigen-decomposition step. 
\end{remark}
\begin{remark}
Despite the important role that $\gamma$ plays in the efficiency of HK algorithm, it seems that the behavior of the parameter $\gamma$ is difficult to analyze. Even a polynomial(in the dimension $d$) lower bound of $\gamma$ is not yet available. 
To overcome this problem, we refine the HK algorithm by a quasi-whitening procedure, inspired by \citep{arora2012provable} and \citep{frieze1996learning}, as shown in the next Subsubsection.
\end{remark}
\begin{remark}
In practice, HK algorithm may generate complex output, even for real input. This occurs only when $\nabla^2\widehat{f}(\psi)$ is singular due to numerical precision. Note that theoretically this happens with probability 0. One way of fixing this problem is sampling a new pair of $\phi$ and $\psi$.
\end{remark}

\subsection{Refined HK algorithm}
\label{subsec:DICA}
The new algorithm, called Determined ICA, is as follows. 
\begin{algorithm}[H]
\caption{Determined ICA}
\begin{algorithmic}[1]
\INPUT $g(k)$ for $1\le k \le T$. 
\OUTPUT An estimation of the mixing matrix $A$. 
\STATE Evaluate $\nabla^2\widehat{f}(\psi)$, \\
%\quad where $\widehat{m_p}(\eta) = \frac{1}{T}\sum_{k=1}^{T} (\eta^{\top}g(k))^p$, and $\widehat{f}(\eta) = \frac{1}{12}\big(\widehat{m_4}(\eta) - 3\widehat{m_2}(\eta)^2 \big)$;
\STATE Compute $\widehat{B}$ such that $\nabla^2\widehat{f}(\psi) = \widehat{B}\widehat{B}^{\top}$;
\STATE Sample $\phi_1$ and $\phi_2$ from the unit sphere of $\real^d$;
\STATE Compute $\widehat{T}_1 = \widehat{G}(\widehat{B}^{-\top}\phi_1)$ and  $\widehat{T}_2 =\widehat{G}(\widehat{B}^{-\top}\phi_2)$;

\STATE Compute all the eigenvectors of $\widehat{T} = \widehat{T}_1\left(\widehat{T}_2\right)^{-1}$, $\{\mu_1,\ldots,\mu_d\}$;
\STATE Return $\widehat{A} = \{\mu_1,\ldots,\mu_d\}$.
\end{algorithmic}
\end{algorithm}

Let 
\begin{align*}
& \bar{\xi} =   \frac{\sqrt{2}d^6A_{(2,\max)}^2A_{\max}^2}{\sqrt{\pi}\kappa_{\min}A^2_{(2,\min)}\sigma_{\min}^2(A)}\xi, \\
& \widehat{\xi} = \frac{16d^9}{\pi^2\kappa_{\min}^2A^4_{(2,\min)}}\xi + 2\bar{\xi},
\end{align*} 
and 
\begin{equation}
\label{def:gammaR}
\gamma_R =  \min_{i,j: i\neq j} \left\vert \left(\frac{\phi_1^{\top}(R^*R)_i}{\phi_2^{\top}(R^*R)_i}\right)^2 - \left(\frac{\phi_1^{\top}(R^*R)_j}{\phi_2^{\top}(R^*R)_j}\right)^2 \right\vert. 
\end{equation}
Similar to Lemma \ref{lem:dmin}, with probability at least $1/4$, $\text{min}_i |\phi_1^{\top}R^*R_i| \ge \frac{\sqrt{2\pi}}{2d}$ (respectively $\text{min}_i |\phi_2^{\top}R^*R_i| \ge \frac{\sqrt{2\pi}}{2d}$). 
Denote this event by $\Ephione$ (respectively $\Ephitwo$).
The performance of the algorithm is guaranteed by the following theorem.
\begin{thm}
\label{thm:Modefficiency}
Let 
 \[ 
 Q=  \frac{32\sqrt{2}d^5\kappa^{3/2}_{\max}A^6_{(2,\max)}}{\pi^{5/2}\kappa_{\min}A^4_{(2,\min)}} \sqrt{\widehat{\xi}}.
 \] 
 Assume the following conditions hold:
 \begin{itemize}
 \item $\widehat{T}$ has distinct eigenvalues;
 \item $\gamma_R > 4\sqrt{2}Q$;
 \item $\xi \le \frac{\sqrt{\pi}\kappa_{\min}A^2_{(2,\min)}\sigma_{\min}^2(A)}{2\sqrt{2}d^6A_{(2,\max)}^2A_{\max}^2}$ (so $\bar{\xi} \le 1/2$);
 \item $\widehat{\xi} \le \frac{\pi}{4d^2\kappa_{\max}A^4_{(2,\max)}}$.
 \end{itemize}
Then on the event $\Epsi \cap \Ephione \cap \Ephitwo$, there exists a permutation matrix $\Pi$ and constants $\{c_1,\ldots,c_d\}$, such that 
\[
\|\widehat{A}\Pi D(c) - A\|_2 
\le 
\sqrt{2} A_{(2,\max)}\sigma_{\max}(A)\kappa_{\max}^{1/2} \left( \bar{\xi} + \frac{4\sqrt{d}}{\gamma_R}Q\right). 
\]
\end{thm}
\subsection{Behavior of $\gamma_R$}
\label{subsec:gammaR}
Even though the behavior of $\gamma$ is unknown, we can calculate a constant-probability  lower bound for $\gamma_R$. 
Note that we can assume $\phi_1$ and $\phi_2$ are independently sampled from standard Gaussian distribution. 
Thus, $\{\phi_1^{\top}R^*R_1, \cdots, \phi_1^{\top}R^*R_d,$ $\phi_2^{\top}R^*R_1, \cdots, \phi_2^{\top}R^*R_d\}$ are $2d$ independent standard normal random variables. 
Let $Z_i = \frac{\phi_1^{\top}(R^*R)_i}{\phi_2^{\top}(R^*R)_i}$. Therefore, $Z_i$, $1\le i\le d$ are $d$ independent Cauchy$(0,1)$ random variables. 

Moreover for any $1\le i, j \le d$, to analyze $\left\vert Z_i^2 - Z_j^2\right\vert
$, WLOG we can assume both $Z_i$ and $Z_j$ are positive. Then, 
\begin{align*}
\gamma_R =	& \min_{i\neq j} \left\vert Z_i^2 - Z_j^2 \right\vert \\
	=		& \min_{i\neq j}\left\vert Z_i - Z_i \right\vert	\left\vert Z_i + Z_i \right\vert \\
	\ge 	& 2\min_i\vert Z_i\vert\min_{i\neq j} \left\vert Z_i - Z_j \right\vert.
\end{align*}
Recall that on the event $\Ephione$, $\min_i\vert Z_i \vert \ge \frac{\sqrt{2\pi}}{2d}$. 
Also note that $\min_{i\neq j} \left\vert Z_i - Z_j \right\vert$ is the distribution of the minimal spacing of Cauchy random variables.

\begin{lemma}
\label{lem:CauchyGap}
With probability at least $1/4$,
\[
\min_{i\neq j} \left\vert Z_i - Z_j \right\vert \ge \frac{\pi}{2d^2}.
\]
\end{lemma}
Thus, by Lemma \ref{lem:CauchyGap}, $\gamma_R \ge \frac{\sqrt{2\pi^3}}{2d^3}$ with probability at least $1/4$.

\section{Experimental Results}
\label{sec:ExpRes}
\subsection{Comparison of $\gamma$ and $\gamma_R$}
Even though we can't develop any theoretical results for $\gamma$, we can compare it with $\gamma_R$ empirically.

\subsection{Simulations}
We investigate the performances of different ICA algorithms in a simulation setting. In particular, four algorithms are tested: 
\begin{itemize}
\item \citet{DHsu2012}'s algorithm(HK);
\item The refined \citet{DHsu2012}'s algorithm (DICA);
\item The default FastICA algorithm in the 'ITE' toolbox \cite{szabo12separation} (FICA). 
\end{itemize}
\subsubsection{Data generation}
In the simulation, a common mixing matrix $A$ is generated, then we sample $k$ groups of $n$ observations. 
For each observation $y$, $y = Ax+ 0.3\times\eps$ where the signal $x$ is generated from a product measure of $d$ uniform distributions $\text{Unif}(-\frac12, \frac12)$, and $\eps$ is generated from a standard $d$-dimensional Gaussian distribution. 
All the algorithms will be evaluated on these common $k$ group of observations.
\subsubsection{Error measures}
We measure the performances of the algorithms on three different measures for different purposes.

The first measure is the parameter recover error. in particular, we evaluate the following quantity between the true mixing matrix $A$ and the one returned by the algorithms $\widehat{A}$:
\begin{equation}
\label{equ:parerror}
\min_{\Pi,S} \|\widehat{A}\Pi S - A\|_{\text{Frob}},
\end{equation}
where $\Pi$ is a permutation matrix, and $S$ is a column scaling matrix (diagonal).

The second measure we used is an approximation of Equation \eqref{equ:parerror} proposed by \citet{comon1994independent}. 
Note that to evaluate Equation \eqref{equ:parerror} one has to enumerate all the permutation $\Pi$, which is not affordable in practice for a large dimension $d$. 
This error helps avoid this computation problem. 

The last measure is the divergence between the joint distribution of the recovered signals and the product of its marginal ones. 
This measure is common in practice when the true mixing matrix $A$ is unknown. 
Moreover, this measure can be reduced to the sum of the entropies of its marginal distribution, as shown in \cite{Learned-Miller:2003:IUS:945365.964306}. 
In particular for an output $\widehat{A}$, we evaluate the following quantity:
\begin{equation}
\sum_{i = 1}^{d} \text{Entropy}(\widehat{x_i}) + \log |\widehat{A}|,
\end{equation}
where $\widehat{x} = \widehat{A}^{-1}y$, and $|\widehat{A}|$ is the absolute value of the determine of $\widehat{A}$. We also use the entropy estimation function 'HShannon\_kNN\_k\_estimation' in the 'ITE' toolbox \cite{szabo14information}.

%For each algorithm, we also measure its running time. 
\subsubsection{Results}


\section{Conclusion}

\if0
\begin{figure}[h]
\vspace{.3in}
\centerline{\fbox{This figure intentionally left non-blank}}
\vspace{.3in}
\caption{Sample Figure Caption}
\end{figure}


\begin{table}[h]
\caption{Sample Table Title} \label{sample-table}
\begin{center}
\begin{tabular}{ll}
{\bf PART}  &{\bf DESCRIPTION} \\
\hline \\
Dendrite         &Input terminal \\
Axon             &Output terminal \\
Soma             &Cell body (contains cell nucleus) \\
\end{tabular}
\end{center}
\end{table}
\fi

\subsubsection*{Acknowledgements}


\bibliography{DICA}
\bibliographystyle{plainnat}
\end{document}
